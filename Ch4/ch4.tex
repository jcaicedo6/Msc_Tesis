\chapter{Posibles fuentes de errores sistemáticos}
En el momento en que una medición como la realizada en la presente tesis se aplica a datos producidos por el experimento se deben considerar errores que van más allá de las incertidumbres estadísticas. No hay una única manera de calcular las incertidumbres sistemáticas en un experimento, pero la experiencia recavada por investigadores y la repetición de ciertas medidas a lo largo de tiempo ha proporcionado que se logren identificar fuentes principales de este tipo de errores, como lo son, diseño del experimento, teniendo en cuenta los propios componentes del mismo, la eficiencia, la calibración y resolución.

También es importante resaltar que la manera en cómo se lleva a cabo el proceso de medición ocasiona la aparición de incertidumbres sistemáticas, como en nuestro caso, vemos que todos los métodos basados en un ``edge'' o ``borde'' como lo llamamos en este trabajo tienen intrínsecamente un sesgo. Otras posibles fuentes de sistemáticos pueden ser los modelos teóricos, los ruidos, simulación MC y hasta el investigador mismo.

Para en nuestro caso podemos tener las siguientes posibles fuentes adicionales de incertidumbres sistemáticas

\begin{itemize}
    \item \textbf{Sistemático debido a la elección de la función de ajuste}.\\
    Considerando la arbitrariedad en la elección de la PDF para ajustar a los datos es necesario analizar la variación en la medición de la masa usando diferentes funciones de ajuste, para este caso se debe calcular la varianza de las diferentes masas halladas, el resultado de esto se asocia con el sistemático
    \item \textbf{Sistemático debido al rango del ajuste}.\\
    Cuando  se escogen diferentes rangos de ajuste, el valor del estimador cambiará, esto hace que se deba considerar una incertidumbre respecto a la medición de la masa.
    \item \textbf{Sistemático asociado a la incertidumbre en la energía del haz}. \\
    Desde la reconstrucción completa de mesones B, teniendo en cuenta la incertidumbre en la masa del B, la calibración en la reconstrucción de las trayectorias y el efecto del ancho en la resonancia \(\Upsilon(4S)\) se calcula la energía del haz. Viendo cómo se propaga el error del anterior cálculo en los datos de MC se puede lograr hallar el sistemático debido a la energía del haz.
    \item \textbf{Sistemático asociado a productos del decaimiento del \(\tau\) mal identificados y eventos no-\(\tau^+\tau^-\)}.\\
    Se debe buscar estructura en la región de ajuste  que pueda afectar la medición de la masa del tau.   
    
\end{itemize}