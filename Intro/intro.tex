\chapter{Introducción}

En los intentos del ser humano para comprender de qué está hecho el universo, se han llegado a avances mayúsculos, como el descubrimiento del Bosón de Higgs en 2012 \cite{Chatrchyan_2012} que da una explicación (aunque aún no del todo completa) al problema anacrónico de la masa, ya que es el campo de Higgs quien dota de masa a los leptones a través de la interacción de Yukawa, esto según el Modelo Estándar de Partículas Elementales (SM), pero aún continúa siendo un misterio qué mecanismo le da masa al mismo bosón de Higgs o también a los neutrinos, que se ha comprobado que tienen una masa debido a su propia oscilación, esto descubierto por Takaaki Kajita y Arthur B. McDonald. Lo anterior nos indica que hay indicios que el SM no está completo o que debe haber física más allá del modelo estándar (BSM), es por esto que resulta interesante y de alta relevancia estudiar propiedades de las diferentes partículas elementales.


En miras de proseguir en el entendimiento del misterio sobre ``¿de qué estamos hechos?'', surge como consecuencia de darle una interpretación a las soluciones de la ecuación de Dirac la existencia de un tipo de materia con algunas propiedades completamente opuestas como lo es la antimateria, i.e, una partícula y su antipartícula (por ejemplo el protón y el antiprotón) tienen la misma masa, pero carga eléctrica de signo opuesto, de igual manera que algunos números cuánticos. 

En la era de la radiación, instantes luego de la gran explosión habían sólo fotones en el universo, a medida que este se fue enfriando tales fotones debieron decaer en pares de partícula-antipartícula, esto nos puede llevar a pensar que en el universo debe haber la misma cantidad de ambas sustancias (materia y antimateria), pero si es así, ¿por qué lo que conocemos y hasta nosotros mismos estamos hechos de materia ordinaria y no de su contraparte?. Este problema recibe el nombre de bariogénesis, donde se reúnen los procesos hipotéticos mediante los cuales se produjo la asimetría entre bariones y antibariones (partículas conformadas por tres quarks) durante los primeros instantes de vida del universo, resultando en cantidades elevadas de materia ordinaria residual en el universo actual. Esta resulta ser una gran motivación para que los grandes experimentos alrededor del mundo, entre ellos Belle, quien con cierto grado de éxito trató de darle respuesta a este gran interrogante encontrando que en el sector de los mesones B existe una diferencia entre materia y antimateria en condiciones similares al comienzo del universo: La llamada violación CP (primero medida en partículas llamadas Kaones en 1964) descrita por el mecanismo de Kobayashi-Maskawa \cite{Kobayashi:1973fv}, trabajo por el cual recibieron el premio Nobel de física en el año 2008 luego de los resultados del experimento anteriormente mencionado. Pero esto no es suficiente, dicha diferencia no es sustancial para dar explicación a la asimetría materia-antimateria, es por esta razón que se han hecho mayores apuestas que se ven reflejadas en una actualización no sólo estructural, sino de software y poder de cómputo, tanto en el acelerador KEKB que ahora pasó a ser SuperKEKB que es un acelerador de leptones de 3016 m de perímetro donde circulan electrones a una energía de 7 GeV en un sentido y positrones a una energía de 4 GeV en el sentido opuesto, estos haces chocan en un punto a una energía del centro de momentos de aproximadamente 10.58 GeV que corresponde a la energía de la resonancia \(\Upsilon (4S)\), ahí, en este punto se encuentra ubicado el detector Belle II (actualización del detector Belle). Entre las mejoras de la actualización se encuentra una mayor luminosidad instantánea, un mejor sistema de triggers, que en conjunto hacen que Belle II pueda tomar datos unas 40 veces más rápido que su predecesor. Al contar con más estadística, también mejores sistemas de trigger y software en general, se prevé que podrían existir las condiciones necesarias para encontrar física BSM, de ser así, podríamos muy prontamente responder el por qué de tal asimetría y otras cuestiones de gran relevancia en la física actual.

Es importante para poder llegar a encontrar resultados de nueva física en el marco del nuevo experimento Belle II primero confirmar o validar resultados del SM, de modo que sea factible llegar a resultados más precisos al realizar mediciones de parámetros y/o procesos bien conocidos. También, poder de esta manera identificar partículas, hallar nuevas o estudiar procesos que contengan ese toque de nueva física tan buscada en la actualidad. Es en este punto donde entra en acción el grupo de física de altas energías (HEP) del CINVESTAV que trabaja en la colaboración internacional Belle II, desde donde en particular se hacen aportaciones en la física del tau. Estudiar el leptón tau es de gran importancia, porque esta partícula es muy sensible a la nueva física debido a algunas características peculiares, entre ellas, que es el leptón más masivo de las tres familias y que tiene una vida corta. Estas características hacen del leptón tau una partícula difícil de detectar y con muchos canales de decaimiento; esta partícula es 1.500 veces más pesada que un electrón, pero tiene la misma carga y los mismos números cuánticos, exceptuando el sabor leptónico.
La gran masa del tau abre la posibilidad de estudiar muchos modos de desintegración cinemáticamente permitidos y extraer información dinámica relevante de allí. Que el leptón tau sea el más masivo de las tres familias y que este sea el único de los leptones que puede decaer en hadrones incentiva que la física del tau sea un laboratorio extraordinario  para estudiar la cromodinámica cuántica (QCD), ya que los hadrones son partículas compuestas, además de ser el leptón que nos proporciona un marco de estudio completo para la electrodinámica cuántica (QED). También en las desintegraciones del leptón tau se podrían encontrar violaciones de las leyes de conservación, como por ejemplo, la violación de carga-paridad (CPV), situación que está muy relacionada con el momento dipolar eléctrico (EDM) del tau. A través de procesos de desintegración que no están considerados dentro del modelo estándar (SM) y que tienen al leptón tau como protagonista se podrían alcanzar los próximos resultados de gran impacto en la comunidad, es por estas razones que la física del tau es de suma importancia en la física de partículas actual. Con estos ambiciosos propósitos en movimiento debemos asegurarnos de que el experimento sea capaz de detectar partículas y también que seamos capaces de reconstruir procesos que son bien conocidos y es por medio de mediciones de diferentes parámetros como la masa de las partículas que esto se logra. Estos procesos son los del SM y mediante estos podemos determinar con mayor precisión parámetros predichos teóricamente y medidos en grandes experimentos previamente. Por ejemplo, la mejor medida de la masa de la tau fue realizada por la colaboración BESIII \cite{PhysRevD.90.012001} (PDG) en 2014, seguidos por BaBar \cite{Aubert_2009} en 2009 y por Belle \cite{PhysRevLett.99.011801} en 2007.


La masa de los quarks y leptones son parámetros fundamentales en el SM, en esta tesis se propone usar un nuevo método para la medición de la masa del leptón tau tomando un canal de decaimiento que hasta el momento no ha sido utilizado en la colaboración Belle II debido a inconvenientes de carácter técnico que se presentan al momento de detectar las partículas de tal decaimiento, ya que tenemos en este partículas invisibles para nuestros detectores, también los métodos hasta ahora usados para medir la masa del tau como el método ARGUS ha usado topologías diferentes \cite{Albrecht1995}. El método de ARGUS es un método de pseudo-masa que considera algunas aproximaciones cinemáticas, como por ejemplo, asumir masa cero del neutrino del tau, aproximar la dirección de vuelo del tau a la de las tres partículas cargadas y tomar la energía del tau como la mitad de la energía del centro de masa. Como cada tau involucra una partícula perdida (neutrino) y además por el tiempo de vida tan corto del tau que lo hace indetectable, esto hace que con el método ARGUS sea imposible la reconstrucción completa del decaimiento y por lo tanto tengamos fuentes de errores sistemáticos importantes. Con esta motivación es que para esta tesis se ha decidido usar un método que originalmente se planteó para medir la masa de partículas invisibles en procesos de decaimiento que pueden estar más allá del modelo estándar y se adaptó a una topología 1x1 prong que pueden suceder en colisiones electrón-positrón \cite{PhysRevD.95.075037}. El método está basado simplemente en solucionar ecuaciones cinemáticas, teniendo en cuenta las variables que los detectores en colisionadores leptónicos pueden entregar, estas son variables de entrada para la ejecución del método. Es de esta manera que se adaptó dicho método para aplicarlo en un decaimiento donde pares de leptones tau son producidos en la colisión electrón-positrón y en particular en procesos donde el tau decae a un pión y un neutrino con la intención extraer la masa del leptón tau. 

La masa del leptón tau es uno de los parámetros fundamentales del SM. La relación entre el tiempo de vida del tau (\(\tau_{\tau}\)), su branching electrónico y el acople débil (\(g_{\tau}\)), viene dada por la teoría
\begin{equation}
    \frac{Br(\tau\rightarrow e\nu\bar{\nu})}{\tau_{\tau}}=\frac{g_{\tau}^2m_{\tau}^5}{192\pi^3}.
\end{equation}

Una medida precisa de la masa del tau es esencial para verificar la universalidad leptónica que es un ingrediente básico para el SM, éste requiere que la intensidad de los acoples sean idénticos, es decir, \(g_{e}=g_{\mu}=g_{\tau}\). La universalidad leptónica puede ser comprobada con la fracción
\begin{equation}
    \left(\frac{g_{\tau}}{g_{\mu}}\right)^2=\frac{\tau_{\mu}}{\tau_{\tau}}\left(\frac{m_{\mu}}{m_{\tau}}\right)^5\frac{Br(\tau\rightarrow e\nu\bar{\nu})}{Br(\mu\rightarrow e\nu\bar{\nu})}(1+F_{W})(1+F_{\gamma}),
\end{equation}
donde \(F_{W}\) y \(F_{\gamma}\) son las correcciones débil y radiativa.

Con el propósito de realizar un proceso de comparación y así obtener el mejor resultado de la medida de la masa del leptón tau con la topología 1x1 prong se ha decidido usar también la técnica de masa mínima y máxima \cite{PhysRevD.102.115001}, donde se asume al neutrino del tau con una masa nula (aproximación usada en el método ARGUS).

En el capítulo 2 de esta tesis se expone de forma breve y descriptiva el modelo fenomenológico que con mayor éxito describe la materia que nos rodea por medio de tres de las cuatro interacciones fundamentales de la naturaleza, este es el Modelo Estándar de Partículas Elementales. Esta teoría es el marco teórico de referencia para la física del leptón tau. La masa del tau es a su vez un parámetro de dicho modelo.

En el capítulo 3 se realiza una descripción del experimento, donde en primera instancia se habla del acelerador SuperKEKB y en segunda instancia del detector Belle II y sus principales componentes.

En el capítulo 4 se realiza la estimación de la masa del leptón tau por tres métodos distintos. Primero se realiza la descripción de los métodos y el modelo matemático de los mismos, también se exponen los criterios de selección de datos utilizados para finalmente se muestran los ajustes realizados para cada método.

En el capítulo 5 se describen muy brevemente las posibles fuentes de errores sistemáticos para cuando estos análisis sean realizados en datos reales del experimento.