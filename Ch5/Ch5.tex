\chapter{Conclusiones}
\begin{itemize}
    \item En este trabajo se adaptaron tres métodos para la medición de la masa del leptón tau basados en la solucionabilidad de ecuaciones cinemáticas. El primero de ellos para una región completa de solución que denominamos ``\(M_{borde}\)'', en este se observó que la densidad de eventos tiende a ser mayor cerca al punto de masa verdadera. Los otros dos métodos ``\(M_{min}\)'' y ``\(M_{max}\)'' tuvieron como aproximación la consideración de masa nula para el neutrino del tau. Como resultado principal de esta tesis se optine que el mejor método para la medición de la masa del leptón tau en la topología \textit{1x1-prong} (que aún no ha sido implementada en la colaboración Belle II) es el método ``\(M_{min}\)'', donde se obtiene un valor para la masa del leptón tau de \(m_{\tau}=1777.06\pm0.47\) MeV. Este resultado se obtiene con los datos oficiales de Belle II simulados en la campaña MC13a.
    \item Los métodos para realizar mediciones de masa en decaimientos cuyos productos finales se clasifican como semi-invisibles usados en esta tesis naturalmente poseen un sesgo debido a la imposibilidad de reconstruir en su totalidad dichos decaimientos. El método que presentó un mayor sesgo fue ``\(M_{borde}\)'' con un valor de \(\Delta_{m}=11.2.\) MeV.
\end{itemize}
