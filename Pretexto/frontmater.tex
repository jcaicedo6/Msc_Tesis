
\newpage


\chapter{Resumen}

En esta tesis se determina la masa del leptón tau por medio de la comparación de tres métodos para decaimientos con estados finales semi-invisibles.
La medición fue realizada para el decaimiento \(\tau\rightarrow\pi\nu\) (señal). Se obtiene que el mejor resultado fue brindado por el método \(M_{min}\) para una masa del leptón tau de \(m_{\tau}=1777.06\pm0.44\) MeV. La medición se completó usando datos oficiales de Monte Carlo de la colaboración Belle II en la campaña MC13a finalizada en el año 2020.

  
